\documentclass[a4paper,11pt,draft]{article}
% !TEX spellcheck = en_UK

%Packages
\usepackage[T1]{fontenc} % fonts have 256 characters instead of 128
\usepackage[utf8]{inputenc} % input file can use UTF8, e.g. ö is recognised
\usepackage[english]{babel}
\usepackage{geometry}

\usepackage{amsmath,amsfonts,amssymb,amsthm,mathrsfs}
\usepackage[hidelinks]{hyperref}
\usepackage{cleveref}
\usepackage{paralist}
\usepackage{enumitem}
\usepackage{textcase} % defines '\MakeTextUppercase'
\usepackage{filemod} % '\moddate' returns the last mod date of the file
\usepackage[longnamesfirst]{natbib}
\usepackage{titlesec} % Used to redefine format of section heading
\usepackage{etoolbox} % Use '\AfterEndEnvironment' to avoid indent after theorems

\usepackage{showlabels}

\makeatletter

%set the default cite command
\def\cite{\citet}

% AMS Math Setup

\numberwithin{equation}{section}
\allowdisplaybreaks

% After theorems and proofs don't indent
\def\@noindentfalse{\global\let\if@noindent\iffalse}
\def\@noindenttrue {\global\let\if@noindent\iftrue}
\def\@aftertheorem{%
  \@noindenttrue
  \everypar{%
    \if@noindent%
      \@noindentfalse\clubpenalty\@M\setbox\z@\lastbox%
    \else%
      \clubpenalty \@clubpenalty\everypar{}%
    \fi}}

\theoremstyle{plain}
\newtheorem{theorem}{Theorem}[section]
\AfterEndEnvironment{theorem}{\@aftertheorem}
\newtheorem{definition}[theorem]{Definition}
\AfterEndEnvironment{definition}{\@aftertheorem}
\newtheorem{lemma}[theorem]{Lemma}
\AfterEndEnvironment{lemma}{\@aftertheorem}
\newtheorem{corollary}[theorem]{Corollary}
\AfterEndEnvironment{corollary}{\@aftertheorem}

\theoremstyle{definition}

\newtheorem{example}[theorem]{Example}
\AfterEndEnvironment{example}{\@aftertheorem}
    
\AfterEndEnvironment{proof}{\@aftertheorem}

\theoremstyle{remark}
\newtheorem{remark}{Remark}
\AfterEndEnvironment{remark}{\@aftertheorem}

% Redefine section headers
\titleformat{name=\section}
  {\center\small\sc}{\thesection\kern1em}{0pt}{\MakeTextUppercase}
\titleformat{name=\section, numberless}
  {\center\small\sc}{}{0pt}{\MakeTextUppercase}

\titleformat{name=\subsection}
  {\bf\mathversion{bold}}{\thesubsection\kern1em}{0pt}{}
\titleformat{name=\subsection, numberless}
  {\bf\mathversion{bold}}{}{0pt}{}

\titleformat{name=\subsubsection}
  {\normalsize\itshape}{\thesubsubsection\kern1em}{0pt}{}
\titleformat{name=\subsubsection, numberless}
  {\normalsize\itshape}{}{0pt}{}



%%% Define bracket commands
\usepackage{delimset}
\def\given{\mskip 0.5mu plus 0.25mu\vert\mskip 0.5mu plus 0.15mu}
\newcounter{bracketlevel}% 
\def\@bracketfactory#1#2#3#4#5#6{%
\expandafter\def\csname#1\endcsname##1{%
\global\advance\c@bracketlevel 1\relax%
\global\expandafter\let\csname @middummy\alph{bracketlevel}\endcsname\given%
\global\def\given{\mskip#5\csname#4\endcsname\vert\mskip#6}\csname#4l\endcsname#2##1\csname#4r\endcsname#3%
\global\expandafter\let\expandafter\given\csname @middummy\alph{bracketlevel}\endcsname%
\global\advance\c@bracketlevel -1\relax%
}%
}
\def\bracketfactory#1#2#3{%
\@bracketfactory{#1}{#2}{#3}{relax}{0.5mu plus 0.25mu}{0.5mu plus 0.15mu}
\@bracketfactory{b#1}{#2}{#3}{big}{1mu plus 0.25mu minus 0.25mu}{0.6mu plus 0.15mu minus 0.15mu}
\@bracketfactory{bb#1}{#2}{#3}{Big}{2.4mu plus 0.8mu minus 0.8mu}{1.8mu plus 0.6mu minus 0.6mu}
\@bracketfactory{bbb#1}{#2}{#3}{bigg}{3.2mu plus 1mu minus 1mu}{2.4mu plus 0.75mu minus 0.75mu}
\@bracketfactory{bbbb#1}{#2}{#3}{Bigg}{4mu plus 1mu minus 1mu}{3mu plus 0.75mu minus 0.75mu}
}
\bracketfactory{clc}{\lbrace}{\rbrace}
\bracketfactory{clr}{(}{)}
\bracketfactory{cls}{[}{]}
\bracketfactory{abs}{\lvert}{\rvert}
\bracketfactory{norm}{\lVert}{\rVert}
\bracketfactory{floor}{\lfloor}{\rfloor}
\bracketfactory{ceil}{\lceil}{\rceil}
\bracketfactory{angle}{\langle}{\rangle}

% Fix spacing problems with \left and \right 
% Note: ()^2 is now typeset properly
\let\original@left\left
\let\original@right\right
\renewcommand{\left}{\mathopen{}\mathclose\bgroup\original@left}
\renewcommand{\right}{\aftergroup\egroup\original@right}


% Define font commands
\newcounter{ctr}\loop\stepcounter{ctr}\edef\X{\@Alph\c@ctr}%
	\expandafter\edef\csname s\X\endcsname{\noexpand\mathscr{\X}}
	\expandafter\edef\csname c\X\endcsname{\noexpand\mathcal{\X}}
	\expandafter\edef\csname b\X\endcsname{\noexpand\boldsymbol{\X}}
	\expandafter\edef\csname I\X\endcsname{\noexpand\mathbb{\X}}
\ifnum\thectr<26\repeat

% Define macros
\let\@IE\IE\let\IE\undefined
\newcommand{\IE}{\mathop{{}\@IE}\mathopen{}}
\let\@IP\IP\let\IP\undefined
\newcommand{\IP}{\mathop{{}\@IP}}
\newcommand{\Var}{\mathop{\mathrm{Var}}}
\newcommand{\Cov}{\mathop{\mathrm{Cov}}}
\newcommand{\bigo}{\mathop{{}\mathrm{O}}\mathopen{}}
\newcommand{\lito}{\mathop{{}\mathrm{o}}\mathopen{}}
\newcommand{\law}{\mathop{{}\sL}\mathopen{}}

% Define accents
\def\^#1{\relax\ifmmode {\mathaccent"705E #1} \else {\accent94 #1} \fi}
\def\~#1{\relax\ifmmode {\mathaccent"707E #1} \else {\accent"7E #1} \fi}
\def\*#1{\relax#1^\ast}
\edef\-#1{\relax\noexpand\ifmmode {\noexpand\bar{#1}} \noexpand\else \-#1\noexpand\fi}
\def\>#1{\vec{#1}}
\def\.#1{\dot{#1}}
\def\wh#1{\widehat{#1}}
\def\wt#1{\widetilde{#1}}
\def\atop{\@@atop}


\renewcommand{\leq}{\leqslant}
\renewcommand{\geq}{\geqslant}
\renewcommand{\phi}{\varphi}
\newcommand{\eps}{\varepsilon}
\newcommand{\D}{\Delta}
\newcommand{\N}{\mathop{{}\mathrm{N}}}
\newcommand{\eq}{\eqref}
\newcommand{\I}{\mathop{{}\mathrm{I}}\mathopen{}}
\newcommand{\dtv}{\mathop{d_{\mathrm{TV}}}\mathopen{}}
\newcommand{\dloc}{\mathop{d_{\mathrm{loc}}}\mathopen{}}
\newcommand{\dw}{\mathop{d_{\mathrm{W}}}\mathopen{}}
\newcommand{\dk}{\mathop{d_{\mathrm{K}}}\mathopen{}}
\newcommand\indep{\protect\mathpalette{\protect\@indep}{\perp}}
\def\@indep#1#2{\mathrel{\rlap{$#1#2$}\mkern2mu{#1#2}}}
\newcommand{\hence}{\mathrel{\Rightarrow}}
\newcommand{\toinf}{\to\infty}
\newcommand{\tozero}{\to0}
\newcommand{\longto}{\longrightarrow}
\def\tsfrac#1#2{{\textstyle\frac{#1}{#2}}}
\newcommand{\hsprod}[1]{\langle #1 \rangle_{\mathrm{H.S.}}}
\DeclareMathOperator{\Tr}{Tr}
\DeclareMathOperator{\Hess}{Hess}
% define \moddate
\def\parsetime#1#2#3#4#5#6{#1#2:#3#4}
\def\parsedate#1:20#2#3#4#5#6#7#8+#9\empty{20#2#3-#4#5-#6#7 \parsetime #8}
\def\moddate{\expandafter\parsedate\pdffilemoddate{\jobname.tex}\empty}
\makeatother


\newlist{ldcondition}{enumerate}{10}
\setlist[ldcondition]{label*=({LD}\arabic*)}
\crefname{ldconditioni}{}{}
\Crefname{ldconditioni}{}{}
%% -------------------------
%% Settings for the cleveref
%% -------------------------
\newcommand{\crefrangeconjunction}{--}
\crefname{equation}{}{}
\crefformat{equation}{(#2#1#3)}
\crefrangeformat{equation}{(#3#1#4)--(#5#2#6)}

\crefname{page}{p.}{pp.}



%%%%%%%%%%%%%%%%%%%%%%%%%%%%%%%%%%%%%%%%%%%%%%%%%%
% Redefinition 
%%%%%%%%%%%%%%%%%%%%%%%%%%%%%%%%%%%%%%%%%%%%%%%%%%
\newenvironment{equ}
{\begin{equation} \begin{aligned}}
{\end{aligned} \end{equation}}
\AfterEndEnvironment{equ}{\noindent\ignorespaces}




\begin{document}

\title{\sc\bf\large\MakeUppercase{nonnormal}}
\author{\sc Songhao Liu}

\date{\itshape CUHK}

\maketitle
\section{Stein's equation} 
Suppose the support of $w$ is $[a,b],$ define
\begin{equation}\label{s6}
  q(t)=\frac{g(t)}{v(t)},
\end{equation}
\begin{equation}\label{s3}
  Q(x)=\int_{k}^{x}q(t)dt
\end{equation}
and 
\begin{equation}\label{s4}
  p(x)=\frac{1}{c_kv(x)}\exp(-Q(x)),
\end{equation}
where $c_k$ is a normalizing constant that makes the above density be a probability density.\\ 

\indent Similar to \cite{Shao2016}, we use the following stein's equation
\begin{equation}\label{s1}
	v(w)f_{z}'(w)-g(w)f_{z}(w)=I_{\{w\leq z\}}-F(z),
\end{equation}
where $F(z)$ is the d.f. of $\chi^2_{k}$. The solution of this equation is
\begin{equation}
\begin{aligned}
	f_{z}(w)&=\frac{1}{v(w)p(w)}\int_{0}^{w}(I_{\{t\leq z\}}-F(z))p(t)dt\\
&=\left\{
      \begin{array}{ll}
        \frac{1}{v(w)p(w)}F(w)(1-F(z))& \hbox{if $0\leq w\leq z$;} \\
        \frac{1}{v(w)p(w)}F(z)(1-F(w))& \hbox{if $z< w $.} \\
      \end{array}
    \right.
\end{aligned}
\end{equation}
From \cite{Shao2016}, we can easily verify that $g(w)$ and $v(w)$ satisfy the condition (B1-B4) in
this paper, so we know that there exist some constant $C_{1,k}$ and $C_{2,k}$ such that $|f_{z}|_\infty\leq C_{1,k}$
and $|f_{z}g|_\infty\leq C_{2,k}$.  
\section{Proof of the results}
\subsection{Properties of the solution}
Now, we will prove some properties of
$f_{z}'(w)=\frac{I_{\{x\leq z\}}-F(z)+g(x)f_{z}(x)}{v(x)}$\\
\indent Since $f_{z} g\leq C_{2,k}$, it is easy to see that for $x> 0$
\begin{equation}
	\begin{aligned}
		\vert f_z'(x)\vert\leq \frac{1+C_{2,k}}{2x}  
	\end{aligned}
	\label{eq:3_7_1}
\end{equation}
\indent To have the monotonicity of $f_{z}'(x)$, we need following conditions
\begin{itemize}
	\item $M_{1}$
		\begin{equation}
			\begin{aligned}
				\lim_{x\to a+}
				\frac{q(x)p(x)}{q'(x)+q^{2}(x)}-\frac{v'(x)p(x)}{v(x)(q'(x)+q^{2}(x))}
				=	\lim_{x\to a+}\frac{v(x) p(x)(g(x)-v'(x))}{v(x)g'(x)-g(x)v'(x)+g^{2}(x)}\geq 0
			\end{aligned}
			\label{eq:5_16_1}
		\end{equation}
	\begin{equation}
			\begin{aligned}
				\lim_{x\to b-}
				\frac{q(x)p(x)}{q'(x)+q^{2}(x)}-\frac{v'(x)p(x)}{v(x)(q'(x)+q^{2}(x))}
				=	\lim_{x\to b-}\frac{v(x) p(x)(g(x)-v'(x))}{v(x)g'(x)-g(x)v'(x)+g^{2}(x)}\leq 0
			\end{aligned}
			\label{eq:5_16_6}
		\end{equation}

	\item $M_{2}$
		\begin{equation}
			\begin{aligned}
				q'(x)+q^{2}(x)=\frac{v(x)g'(x)-g(x)v'(x)+g^{2}(x)}{v^{2}(x)}>0
			\end{aligned}
			\label{eq:5_16_2}
		\end{equation}
	\item $M_{3}$
		\begin{equation}
			\begin{aligned}
				(-g(x)+v'(x))g''(x)+2(g'(x))^{2}-g'(x)v''(x)\geq 0
			\end{aligned}
			\label{eq:5_16}
		\end{equation}
\end{itemize}
\indent For monotonicity of $f_{z}'$ we have the following lemmas.
\begin{lemma}\label{lem4}
	If $f_{z}$ is the solution of (\ref{s1}), under condition $M_{1}-M_{3}$ we
	have $f_z'(x)$ is increasing in the interval $[a,z]$ and interval $(z,b]$ respectively.
\end{lemma}
\proof
\indent For $x\leq z$,
\begin{equation}\label{p2}
\begin{aligned}
	\frac{1-F(z)+g(x)f_{z}(x)}{v(x)}=(1-F(z))\left(
	\frac{1}{v(x)}+q(x)\frac{F(x)}{v(x)p(x)}\right)=(1-F(z)) h_1(x),
\end{aligned}
\end{equation}
where $h_1(x) = \frac{1}{v(x)}+q(x)\frac{F(x)}{v(x)p(x)}$.
\begin{equation}\label{p3}
\begin{aligned}
	h_1'(x)=\frac{1}{v(x)}\left( q(x)+(q'(x)+q^2(x))\frac{F(x)}{p(x)}-\frac{v'(x)}{v(x)}\right). 
\end{aligned}
\end{equation}

% By condition $M_{2}$, $q'(x)+q^2(x)>0$, $p(x)>0$ and
% $\lim\limits_{x\rightarrow0^{+}}\frac{p(x)}{q'(x)+q^{2}(x)}=0$, we have 
% \begin{equation}\label{p5}
% \begin{aligned}
% \lim\limits_{x\rightarrow0^+}\frac{q(x)p(x)}{q'(x)+q^2(x)}+F(x)-\frac{p(x)}{x(q'(x)+q^2(x))}= 0.
% \end{aligned}
% \end{equation}
By condition $M_{1}$
\begin{equation}\label{5_16_4}
\begin{aligned}
	\lim_{x\to a} \frac{q(x)p(x)}{q'(x)+q^2(x)}+F(x)-\frac{p(x)}{x(q'(x)+q^2(x))}=0
\end{aligned}
\end{equation}
By some calculation and by condition $M_{3}$, we have 
\begin{equation}\label{p6}
\begin{aligned}
&\left(\frac{q(x)p(x)}{q'(x)+q^2(x)}+F(x)-\frac{v'(x)p(x)}{v(x)(q'(x)+q^2(x))}
	\right)'\\=& \frac{p(x)}{v^{2}(x) (q'(x)+q^{2}(x))^{2}}  \biggl(
(-g(x)+v'(x))g''(x)+2(g'(x))^{2}-g'(x)v''(x)  \biggr)\geq 0
\end{aligned}
\end{equation}
By condition $M_{2}$, $p(x)\geq 0$ and $v(x)\geq 0$, we have $h_1'(x)\geq 0$ for $x\geq a$, and
$f_z'(x)=(1-F(z))h_1(x)$ is increasing for $x\leq z.$\\ 
	\indent Similarly, for $x> z$,
\begin{equation}\label{p2_2}
\begin{aligned}
	\frac{-F(z)+g(x)f_{z}(x)}{v(x)}=F(z)\left(
	-\frac{1}{v(x)}+q(x)\frac{1-F(x)}{v(x)p(x)}\right)=F(z) h_2(x) .
\end{aligned}
\end{equation}
\begin{equation}\label{p3_2}
\begin{aligned}
	h_2'(x)=\frac{1}{v(x)}\left( -q(x)+(q'(x)+q^2(x))
\frac{1-F(x)}{p(x)}+\frac{1}{x}\right). 
\end{aligned}
\end{equation}
By condition $M_{1}$, we have 
\begin{equation}\label{p5_2}
\begin{aligned}
\lim\limits_{x\rightarrow b}\frac{-q(x)p(x)}{q'(x)+q^2(x)}+1-F(x)+\frac{p(x)}{x(q'(x)+q^2(x))}= 0.
\end{aligned}
\end{equation}
By some calculation and by $M_{3}$, we have 
\begin{equation}\label{p6_2}
\begin{aligned}
&\left(\frac{-q(x)p(x)}{q'(x)+q^2(x)}+1-F(x)+\frac{p(x)}{x(q'(x)+q^2(x))}
\right)' \\=& -\frac{p(x)}{v^{2}(x) (q'(x)+q^{2}(x))^{2}}  \biggl(
(-g(x)+v'(x))g''(x)+2(g'(x))^{2}-g'(x)v''(x)  \biggr)\leq 0.
\end{aligned}
\end{equation}
So $h_2'(x) \geq 0$ for $x>0$, and
$\frac{I_{\{x\leq z\}}-F(z)+g(x)f_{z}(x)}{v(x)}$ is increasing for $x> z.$
\endproof

Combining the monotonicity of $\frac{I_{\{x\leq z\}}-F(z)+g(x)f_{z}(x)}{v(x)}$ and  continuity of
$f_{z}(x)g(x)$, we can easily have 
\begin{equation}\label{p7}
\begin{aligned}
	f_{z}'(x)-\frac{I_{\{x\leq z\}}}{2z}=
	\frac{I_{\{x\leq z\}}-F(z)+g(x)f_{z}(x)}{v(x)}-\frac{I_{\{x\leq z\}}}{2z} 
\end{aligned}
\end{equation}
is increasing for all $x\geq 0. $\\
\indent Next, we are going to consider the property of $f'(x)$ near zero. 
\begin{equation}\label{p8}
\begin{aligned}
	f_{z}'(x)=\frac{I_{\{x\leq z\}}-F(z)+g(x)f_{z}(x)}{v(x)} .
\end{aligned}
\end{equation}
It easy to verify that 
\begin{equation}\label{p9}
	\lim\limits_{x\rightarrow0+}I_{\{x\leq z\}}-F(z)+g(x)f_{z}(x)=0
\end{equation}
for $z>0$. Now we can use L'H\^{o}pital's rule to calculate
$\lim\limits_{x\rightarrow0+}f_{z}'(x),$
\begin{equation}\label{p10}
\begin{aligned}
&\lim\limits_{x\rightarrow0+}\frac{I_{\{x\leq z\}}-F(x)+g(x)f_{z}(x)}{v(x)}\\=&(1-F(z))\lim\limits_{x\rightarrow0+}\frac{F(x)+\frac{p(x)v(x)}{g(x)}}{\frac{p(x)v^2(x)}{g(x)}}\\
	=&(1-F(z))\lim\limits_{x\rightarrow0+}\frac{\left( F(x)+\frac{p(x)v(x)}{g(x)}\right)' }{\left( \frac{p(x)v^2(x)}{g(x)}\right) '}\\
	=&(1-F(z))\lim\limits_{x\rightarrow0+}\frac{1}{k^2-2kx+x^2+2k}\\
	=&(1-F(z))\frac{1}{k^2+2k}.
\end{aligned}
\end{equation}

 \setlength{\bibsep}{0.5ex}
 \def\bibfont{\small}
%\bibliographystyle{unsrt}
\bibliographystyle{plainnat}
%\bibliographystyle{IEEEtranSN}
%\bibliographystyle{mynatbib}
%\bibliography{literature}
\bibliography{reference}
% \begin{thebibliography}{1}

% \bibitem[Knuth(1997)]{Knuth1997a}
% D.~E. Knuth (1997).
% \newblock \emph{The Art of Computer Programming. Vol. 1: Fundamental
%   Algorithms}.
% \newblock Addison-Wesley, 3rd edition.

% \end{thebibliography}


\end{document}

